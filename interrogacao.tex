%%%%%%%%%%%%%%%%%%%%%%%%%%%%%%%%%%%%%%%%%%%%%%%%%%%%
%												   %
%	ESCOLA   									   %
%												   %
%	Maio 2015  									   %
%												   %
%	Angela Cardoso, Rui Costa e Ricardo Lopes	   %
%   											   %	
%%%%%%%%%%%%%%%%%%%%%%%%%%%%%%%%%%%%%%%%%%%%%%%%%%%%

\documentclass[12pt,a4paper,reqno]{report}
\linespread{1.5}

\usepackage{amsfonts,amsmath,amssymb,indentfirst,mathrsfs,amscd}
\usepackage[mathscr]{eucal}
\usepackage[active]{srcltx} %inverse search
\usepackage{tensor}
\usepackage[utf8x]{inputenc}
\usepackage[portuges]{babel}
\usepackage[T1]{fontenc}
\usepackage{tikz}
\usepackage{graphicx}
\usepackage[numbers,square, comma, sort&compress]{natbib}
\numberwithin{figure}{section}
\numberwithin{equation}{section}
\usepackage{scalefnt}
\usepackage[top=2.5cm, bottom=2.5cm, left=2.5cm, right=2.5cm]{geometry}
\usepackage{comment} 
%\usepackage{tweaklist}
%\renewcommand{\itemhook}{\setlength{\topsep}{0pt}%
%	\setlength{\itemsep}{0pt}}
%\renewcommand{\enumhook}{\setlength{\topsep}{0pt}%
%	\setlength{\itemsep}{0pt}}
%\usepackage[colorlinks]{hyperref}
\usepackage{MnSymbol}
%\usepackage[pdfpagelabels,pagebackref,hypertexnames=true,plainpages=false,naturalnames]{hyperref}
\usepackage[naturalnames]{hyperref}
\usepackage{enumitem}
\usepackage{titling}
\newcommand{\subtitle}[1]{%
  \posttitle{%
    \par\end{center}
    \begin{center}\large#1\end{center}
    \vskip0.5em}%
}
\newcommand{\HRule}{\rule{\linewidth}{0.5mm}}

\usepackage[official]{eurosym}

\def\Cpp{C\raisebox{0.5ex}{\tiny\textbf{++}}}

\makeatletter
\def\@makechapterhead#1{%
  %%%%\vspace*{50\p@}% %%% removed!
  {\parindent \z@ \raggedright \normalfont
    \ifnum \c@secnumdepth >\m@ne
        \huge\bfseries \@chapapp\space \thechapter
        \par\nobreak
        \vskip 20\p@
    \fi
    \interlinepenalty\@M
    \Huge \bfseries #1\par\nobreak
    \vskip 40\p@
  }}
\def\@makeschapterhead#1{%
  %%%%%\vspace*{50\p@}% %%% removed!
  {\parindent \z@ \raggedright
    \normalfont
    \interlinepenalty\@M
    \Huge \bfseries  #1\par\nobreak
    \vskip 40\p@
  }}
\makeatother


\begin{document}

INTERROGAÇÕES

\begin{enumerate}
	
	\item Quantos alunos, docentes e encarregados de educação tem a escola?
	
	\item Que docentes são simultaneamente encarregados de educação?
	
	\item Que alunos vivem na mesma localidade que o seu atual diretor de turma?
	
	\item Qual o grau de parentesco mais frequente entre os alunos e os respetivos encarregados de educação?
	
	\item Além de `pai' e `mãe', que outros parentescos existem?
	
	\item Quantos alunos existem em cada Área?
	
	\item Qual é a melhor média da escola e quem são os alunos que têm essa média?
	
	\item Para cada disciplina qual a percentagem de alunos aprovados por período letivo?
	
	\item No final de cada período, cada diretor de turma deve reunir com os encarregados de educação dos alunos que tiraram negativa a alguma disciplina. Para o período mais recente, liste os pares (diretor de turma, encarregado de educação) que devem reunir nestas circunstâncias, incluindo o horário de contacto, o telefone e o telemóvel do encarregado de educação.
	
	\item Quem são os melhores alunos a cada disciplina e qual a sua nota (tendo em consideração apenas notas do final de cada ano letivo)?
	
	\item Para cada Área, qual é o nome, o número de telefone e o número de telemóvel do atual coordenador?
	
	\item Quantos docentes estão aptos a lecionar cada disciplina?
	
	\item Quais as reuniões nas quais os intervenientes têm algum nome em comum?
	
	\item Para cada docente qual a percentagem de alunos aprovados por ano lectivo?
	
	\item Para cada ano letivo e cada ano escolar, quem são os melhores alunos e qual é a sua média?
	
\end{enumerate}

\end{document}
